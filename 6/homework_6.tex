%This is a Latex file.
\documentclass[12pt]{article}
\usepackage{latexsym,fancyhdr,amsmath,amsfonts,amsthm,dsfont}
\usepackage{amssymb}
\usepackage{graphicx} 

% margins are relative to the default of 1 in
%\topmargin       -0.2 in

\topmargin        -0.2 in
\textheight       8.4 in
\oddsidemargin    0 in     % this is for pages 1, 3, 5, ...
\evensidemargin   0 in     % and this for 2, 4, 6, ...
\textwidth        6.5 in
%\headheight       15 in     % we won't have a running head, nor
\headsep          .35 in     % any extra space between head and text

%\parindent 0pt

\pagestyle{fancy} \lhead{\sf MTH 317} \chead{\sf Homework 6}
\rhead{\sf Rayana Gottschall} \lfoot{} \cfoot{} \rfoot{}

\newcommand{\C}{\mathds{C}}
\newcommand{\I}{\mathds{I}}
\newcommand{\N}{\mathds{N}}
\newcommand{\Q}{\mathds{Q}}
\newcommand{\R}{\mathds{R}}
\newcommand{\Z}{\mathds{Z}}

\begin{document}
\begin{enumerate}
\item[2.10] Prove that a connected graph of size at least 2 is non-seperable iff two adjacent edges of G lie on a common cycle.

    \begin{proof}
    ($\Rightarrow$) Assume G is non-seperable. Let e1 and e2 be two adjacent edges of G, with common vertex v. 
    Since G is non-seperable, removing v does not disconnect G. Therefore, there exists a path P from one endpoint of e1 to one endpoint of e2 that does not pass through v. 
    Combining e1, e2, and P forms a cycle containing both e1 and e2.
    
    ($\Leftarrow$) Assume that any two adjacent edges of G lie on a common cycle. Suppose G is separable. 
    Then there exists a cut-vertex v such that G - v creates at least two components. Let e1 and e2 be two edges incident to v. 
    Since e1 and e2 are adjacent they must lie on a common cycle C. Removing v from C would break the common cycle with e1 and e2.
    This contradicts the claim that any two adjacent edges of G lie on a common cycle. 
    Therefore, G must be non-seperable.

    \end{proof}
    

 \item[2.12] If a connected graph G has three blocks and $k$ cut-vertices, what are the possible values of $k$?
    
    $k$ can be 1 or 2. If $k$ is 1, then there is one cut-vertex common to all three blocks. 
    If $k$ is 2, then each cut-vertex is common to two blocks. One of the blocks will contain both cut-vertices.


\item[5.18] Let PG be the Petersen graph. Give an example of
\item[] (a) a minimum vertex-cut
\item[]
    \includegraphics[scale=0.5]{problema.png}
 
\item[] (b) a vertex-cut $U$ s.t. $U$ is not a minimum vertex-cut and no proper subset of $U$ is a vertex-cut
    
\includegraphics[scale=1.2]{Untitled design.png}


\item[5.23 (b)] Prove that if G is a k-edge-connected graph, then G + K is (k + 1)-edge-connected 

    \begin{proof}
        Let G be a k-edge connected graph and let X be an edge-cut of G.
        Since G is k-edge-connected, $|X|$ $\geq$ k. G - X results in two
        components, G1 and G2. Let y $\in$ V(G1) and let z $\in$ V(G2).
        Suppose we have G + K, where K is a new vertex connected to every vertex in G. 
        So G - X + K is connected by the edges yK and zK.
        In the smallest case, suppose G1 only contains y. Then removing yK would disconnect
        G - X + K (there would be no alternitive paths to K from G1). 
        So, to disconnect G + K, we must remove X (containing at least k edges),
        and the edge yk. The total edge-cut becomes k + 1.
        If G1 contains more than one vertex, then we must remove all edges between K and G1,
        in addition to X. Since the edges between K and G1 are at least 1, G is  
        at least (k + 1)-edge-connected.
    \end{proof}

\item[5.24] Let G be a graph of order n and let k be an integer with 1 $\leq$ k $\leq$ n - 1. 
    Prove that if $\delta(G)$ $\geq$ $\frac{n + k - 2}{2}$, then G is k-connected.

    \begin{proof}
       Assume G is not k-connected. Then there exists a vertex-cut S with $|S|$ $\leq$ k - 1 
        such that G - S is disconnected. Let the components of G - S be G1 and G2.
        Let $|V(G1)| = a$ and $|V(G2)| = b$. Then, we have $a + b + |S| = n$.
        Since there are no edges between G1 and G2, each vertex in G1 is only adjacent to vertices in G1 and S.
        So, the degree of any vertex in G1 is at most $a + |S| - 1$. 
        And the degree of any vertex in G2 is at most $b + |S| - 1$.
        Since $\delta(G)$ $\geq$ $\frac{n + k - 2}{2}$, we have:
        \[\frac{n + k - 2}{2} \leq a + |S| - 1 \quad \text{and} \quad \frac{n + k - 2}{2} \leq b + |S| - 1\]
        Rearranging gives:   
        \[a \geq \frac{n + k - 2}{2} - |S| + 1 \quad \text{and} \quad b \geq \frac{n + k - 2}{2} - |S| + 1\]
        We can add these two inequalities:
        \[a + b \geq n + k - 2 - 2|S| + 2\]
        Since $a + b = n - |S|$, we can substitute to get:
        \[n - |S| \geq n + k - 2 - 2|S| + 2\]
        Simplifying:
        \[|S| \geq k\]
        This contradicts the assumption that $|S|$ $\leq$ k - 1. Therefore, G must be k-connected.
        
    \end{proof}

\end{enumerate}
\end{document}