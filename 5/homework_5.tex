%This is a Latex file.
\documentclass[12pt]{article}
\usepackage{latexsym,fancyhdr,amsmath,amsfonts,amsthm,dsfont}
\usepackage{amssymb}

% margins are relative to the default of 1 in
%\topmargin       -0.2 in

\topmargin        -0.2 in
\textheight       8.4 in
\oddsidemargin    0 in     % this is for pages 1, 3, 5, ...
\evensidemargin   0 in     % and this for 2, 4, 6, ...
\textwidth        6.5 in
%\headheight       15 in     % we won't have a running head, nor
\headsep          .35 in     % any extra space between head and text

%\parindent 0pt

\pagestyle{fancy} \lhead{\sf MTH 317} \chead{\sf Homework 5}
\rhead{\sf Rayana Gottschall} \lfoot{} \cfoot{} \rfoot{}

\newcommand{\C}{\mathds{C}}
\newcommand{\I}{\mathds{I}}
\newcommand{\N}{\mathds{N}}
\newcommand{\Q}{\mathds{Q}}
\newcommand{\R}{\mathds{R}}
\newcommand{\Z}{\mathds{Z}}

\begin{document}
\begin{enumerate}
\item[4.8] Prove that if every vertex of a graph G has degree at least 2, then G contains a cycle.

    \begin{proof}
    Assume G does not contain a cycle. If G is connected and acyclic, then G must be a tree. If G is disconnected then it must be a forest. 
    In either case, there exists at least one leaf vertex (a vertex of degree 1). 
    This contradicts the claim that every vertex has degree at least 2. Therefore, G must contain a cycle.
    \end{proof}
    
 \item[4.18] A certain tree $T$ of order n contains only vertices of degree 1 or 3. Show that $T$ has exactly $\displaystyle\frac{n-2}{2}$ vertices of degree 3.
 
    \begin{proof}
    x + 3y = 2(n-1) \newline
    x + 3y = 2(x+y-1) \newline
    x + 3y = 2x + 2y - 2 \newline
    y = x - 2 \newline
    n = x + y = x + x - 2 = 2x - 2  \newline
    n + 2 = 2x   \newline
    x = (n + 2)/2 \newline
    y = (n + 2)/2 - 2 = (n - 2)/2 \newline
    So, T has exactly (n-2)/2 vertices of degree 3. 
    \end{proof}

\item[4.22] Let $T$ be a tree of order n. Show that the size of the complement of $overline{T}$ of $T$ is the same as the size of $K_n-1$.

\begin{proof}
A complete graph $K_n$ has $\binom{n}{2} = \frac{n(n-1)}{2}$ edges.
A tree $T$ with $n$ vertices has $n - 1$ edges.

The complement $\overline{T}$ contains every edge of $K_n$ that is not in $T$, so
\[
|E(\overline{T})| = \binom{n}{2} - (n - 1)
= \frac{n(n-1)}{2} - (n - 1).
\]
Simplifying:
\[
|E(\overline{T})|
= (n - 1)\left(\frac{n}{2} - 1\right)
= \frac{(n-1)(n-2)}{2}
= \binom{n-1}{2}.
\]
Thus, $\overline{T}$ has the same number of edges as $K_{n-1}$.
\end{proof}

\item[5.4] Prove that if v is a cut vertex of G, then v is not a cut vertex of the complement of G.

    \begin{proof}
        Assume v is a cut vertex of $\overline{G}$. 
        After removing v from G, G contains at least two components.
        Let them be U and W such that V(U) $\neq$ V(W). Let x $\in$ V(U) and y $\in$ V(W).
        Since v is a cut vertex of G, there is no edge between x and y in G.
        However, in $\overline{G}$, there must be an edge between x and y.
        Because E(k) - E(G) = E($\overline{G}$), where k is the complete graph.
        So, removing v from $\overline{G}$ does not disconnect the graph. 
        Therefore, the assumption is false and v is not a cut vertex of $\overline{G}$.    
    \end{proof}

\item[5.6]  Prove that a 3-regular graph G has a cut vertex if and only if G has a bridge.

    \begin{proof}
    ($\Rightarrow$) 
    Let \(G\) be a \(3\)-regular graph and suppose \(v\) is a cut vertex of \(G\). \(G-v\) has at least two components, 
    let them be \(H_1,\dots,H_k\) with \(k\ge2\). Every component \(H_i\) contains at least one 
    vertex adjacent to \(v\) since the graph is connected. 
    So the 3 neighbors of \(v\) are distributed among the components. 
    Because \(v\) has degree \(3\), we must have \(2\le k\le 3\). In either case there is at least one component \(H_j\) that contains exactly one 
    neighbour \(u\) of \(v\):  for \(k=2\) the neighbour counts are \(1\) and \(2\), and for \(k=3\) they are \(1,1,1\).
    Let \(e=uv\) be the unique edge joining \(v\) to \(H_j\). Any path from a vertex of \(H_j\) to a vertex outside \(H_j\) must pass 
    through \(v\), and thus must use the edge \(uv\). So deleting \(e\) disconnects \(H_j\) from the rest of the graph, so \(e\) is a bridge of \(G\).
    Therefore a cut vertex in a \(3\)-regular graph must have a bridge.


    ($\Leftarrow$) Assume G does not have a cut-vertex. But G has a bridge, e. G-e is disconnected and contains two components. 
    e must be incident to two vertices, u and v. Since G is 3-regular, both u and v must have two other edges incident to them.
    Since removing e disconnects G, removing u or v would have the same effect. So either u or v must be a cut vertex.
    Since G is 3-regular, we can say there is a cut vertex in G by Theorem 5.1. 
    This contradicts the assumption that G does not have a cut vertex. Therefore, if G has a bridge, then G must have a cut vertex.
    \end{proof}

\end{enumerate}
\end{document}
