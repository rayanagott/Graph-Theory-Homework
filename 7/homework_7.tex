%This is a Latex file.
\documentclass[12pt]{article}
\usepackage{latexsym,fancyhdr,amsmath,amsfonts,amsthm,dsfont}
\usepackage{amssymb}

% margins are relative to the default of 1 in
%\topmargin       -0.2 in

\topmargin        -0.2 in
\textheight       8.4 in
\oddsidemargin    0 in     % this is for pages 1, 3, 5, ...
\evensidemargin   0 in     % and this for 2, 4, 6, ...
\textwidth        6.5 in
%\headheight       15 in     % we won't have a running head, nor
\headsep          .35 in     % any extra space between head and text

%\parindent 0pt

\pagestyle{fancy} \lhead{\sf MTH 317} \chead{\sf Homework 7}
\rhead{\sf Rayana Gottschall} \lfoot{} \cfoot{} \rfoot{}

\newcommand{\C}{\mathds{C}}
\newcommand{\I}{\mathds{I}}
\newcommand{\N}{\mathds{N}}
\newcommand{\Q}{\mathds{Q}}
\newcommand{\R}{\mathds{R}}
\newcommand{\Z}{\mathds{Z}}

\begin{document}
\begin{enumerate}
\item[6.2] Let $G_1$ and $G_2$ be two Eulerian graphs with no vertex in common. 
    Let $v_1 \in V(G_1)$ and $v_2 \in V(G_2)$. 
    Let $G$ be the graph obtained from $G_1 \cup G_2$ by adding the edge $v_1v_2$. 
    What can be said about $G$?\\
    \newline
Since $G_1$ and $G_2$ are Eulerian, every vertex in each graph has even degree.
If we add an additional edge $v_1v_2$ to form $G$, then the degrees of $v_1$ and $v_2$ 
each increase by 1, making them odd. 
Therefore, $G$ has exactly two vertices of odd degree and cannot be Eulerian.

\item[6.6] Let G be a connected regular graph that is not eulerian. Prove that if $\overline{G}$
    is connected, then $\overline{G}$ is eulerian.
    \begin{proof}
    
    \end{proof}
\item[6.8 (a)] Show that every nontrivial graph G has a closed spanning walk that contains every edge
 of G exactly twice.
    
\item[6.10] Let G be a 6-regular graph of order 10 and let $u$, $v$ $\in$ V(G). Prove G - $v$ and G - $u$ - $v$ are all Hamiltonian.
    \begin{proof}
    Let \(G\) be a 6-regular graph of order 10, and let \(u,v \in V(G)\). In \(G - v\), 
    the order is 9. Since \(G\) is 6-regular, removing \(v\) decreases the degree of each 
    of its 6 neighbors by 1, so every vertex in \(G - v\) has degree at least \(5\). Thus 
    \(\delta(G - v) \ge 5\). By Dirac’s Theorem, if a graph \(H\) of order \(n \ge 3\) 
    satisfies \(\delta(H) \ge n/2\), then \(H\) is Hamiltonian. Here, \(n = 9\) and 
    \(\delta(G - v) = 5 \ge 9/2\), so \(G - v\) is Hamiltonian. Now consider \(G - u - v\). 
    The order is 8. Removing \(u\) and \(v\) decreases the degree of each vertex by at most 
    2 (if the vertex was adjacent to both \(u\) and \(v\)), so \(\delta(G - u - v) \ge 6 - 2 
    = 4\). Since \(4 = 8/2\), by Dirac’s Theorem, \(G - u - v\) 
    is also Hamiltonian. Therefore, both \(G - v\) and \(G - u - v\) are Hamiltonian
    \end{proof}

\item[6.12] Let G be a 3-regular graph of order 12 and H a 4-regular graph
of order 11.
\item[(a)] Is G + H Eulerian?
\newline
    Yes. Originally in $G$, $\deg = 3$. In $G + H$, each vertex in $G$ gains an edge 
    to all 11 vertices in $H$, so $\deg = 3 + 11 = 14$.\\
    In $H$, $\deg = 4$, and in $G + H$ each vertex in $H$ gains an edge to all 
    12 vertices in $G$, so $\deg = 4 + 12 = 16$.\\
    Since all vertices have even degree, $G + H$ satisfies the condition for being 
    Eulerian.
\item[(b)] Is G + H Hamiltonian?
\newline
    Yes. By corollary 6.7, a graph of order $n \ge 3$ is Hamiltonian if 
    $\delta(G) \ge \tfrac{n}{2}$.\\
    Here, $n = 12 + 11 = 23$ and $\delta(G + H) = 14$, since 
    $\min(14, 16) = 14$.\\
    Because $14 \ge \tfrac{23}{2}$, the condition holds, so $G + H$ is Hamiltonian

\end{enumerate}


\end{document}
